\def\Oh{$\mathcal{O}$}
\subsection{\bwmstar: Exploiting locality of protein energetics to efficiently compute the GMEC}


{\sc osprey} 3 comes with \bwmstar$^*$~\cite{BWM_JCB}, a new algorithm that exploits sparse energy functions to provably compute the GMEC in time exponential in merely the branch-width $w$ of a protein design problem's sparse residue interaction graph.

Because energy decreases as a function of distance, many protein design algorithms model protein energetics with energy functions which omit pairwise interactions between sufficiently distant residues. These \emph{sparse energy functions} not only provide a simpler, more efficiently computed model of energy, but also induce \emph{optimal substructure} to the problem: because not all residues interact, the optimal conformation for a given residue can be independent of the conformations at other residues. \bwmstar exploits this optimal substructure by 1) representing the sparse interactions with a sparse residue interaction graph, and 2) computing a branch-decomposition for use in dynamic programming. 

\bwmstar enumerates a gap-free list of conformations in order of increasing sparse energy. Because this list is gap-free, \bwmstar not only computes the GMEC of the sparse energy function, but also recovers the GMEC of the full energy function, as shown in~\cite{BWM_JCB}. By enumerating all conformations within the provable sparse energy bound between the sparse and full GMEC, \bwmstar computes a list of conformations which is guaranteed to contain the full GMEC, as well as the sparse GMEC.

Thus, in practice, \bwmstar circumvents the worst-case complexity of traditional methods such as \as for designs with sparse energy functions, computing the sparse GMEC in \Oh$(nw^2q^{\frac{3}{2}w})$ time, but enumerates each additional conformation in merely \Oh$(n\log q)$ time, which is up to three orders of magnitude faster than traditional \as in practice.