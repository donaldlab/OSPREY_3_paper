To evaluate the accuracy of the \osprey \ks algorithm, we performed a series of designs for a variety of protein-protein interfaces as retrospective validation. We used \ks to computationally predict experimentally measured changes in binding for each system. Each protein structure is listed by PDB ID in Table \ref{table:spearman} (find more detailed system information in Supplemental Table X). Mutations reported in the literature were tested for each system 

\anna{Move to supplemental? "These systems include barnase with its peptide inhibitor barstar, the cytochrome {\it c}/cytochrome {\it c} peroxidase complex, interferon (IFN)$\alpha$2 with its receptor, ifnar2, the interleukin 2 (IL-2)/IL-2 receptor $\alpha$ (IL-2R$\alpha$) complex, and an antibody fragment bound to the I-domain of the integrin VLA1."}

Our retrospective validation experiments focused on mutations at residues in or proximal to the protein-protein interface. Including all such experimentally tested and reported mutations, each system ranges from 5 to 36 ranked designs. Each design included one or two mutable residues along with a set of surrounding flexible residues. Flexible residues were chosen by selecting all residues within 4 \AA of the mutable residues and removing those which only have backbone interactions. 

For each system, the \ks scores were ranked in increasing order of experimental binding. Spearman's $\rho$ values were subsequently calculated for each system by calculating the statistical dependence between the \ks score rankings and the experimentally measured rankings. 

\begin{table}[h!]\label{table:spearman}
\centering
\begin{tabular}{ |c||c|  }
 \hline
 \textbf{Structures (PDB ID)}& \textbf{Spearman's $\rho$} \\
 \hline 
 3S9D   & 0.795 \\
 \hline
 1X1U   & 0.755 \\
 \hline
 2B5I   & 0.554 \\
 \hline
 2PCB   & 0.500 \\
 \hline
 1MHP   & 0.365 \\
 \hline 
 \textbf{Across Structures} &   \textbf{0.632}  \\
 \hline
\end{tabular}
\caption{Spearman's $\rho$ table. A Spearman's $\rho$ value is calculated for each system and shown here. The "Across Structures" value is calculated by... }
\end{table}

\begin{figure}\label{fig:rankings}
\center
\includegraphics[width=0.9\textwidth]{figures/Rankings.png}
\caption{Testing the accuracy of the \ks algorithm in \osprey3 by comparing \ks rankings to experimentally reported rankings. Each system is represented by its corresponding PDB ID and a linear trendline is shown for each in its corresponding color according to the legend.}
\end{figure}

