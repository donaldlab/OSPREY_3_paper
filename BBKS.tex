\def\multisequencebound{MS\xspace}
\def\msbound{\multisequencebound}


\newcommand{\cut}[1]{}

In previous versions of \osprey, the \ks algorithm~\cite{K*} modeled an ensemble of Boltzmann-weighted conformations to approximate the thermodynamic partition function. It combined minimized dead-end elimination pruning~\cite{DEE} with \as~\cite{DEE,A*} gap-free conformation enumeration to compute provable $\varepsilon$-approximations to the partition functions for the protein and ligand states of interest. \ks combined these partition function scores to approximate the association constant, \ka, as the ratio of $\varepsilon$-approximate partition functions between the bound and unbound states of a protein-ligand complex. Notably, each partition function ratio, called a \ks \emph{score}, is provably accurate with respect to the biophysical \emph{input model}~\cite{K*,minDEE,iMinDEE}. 

Although \ks efficiently and provably approximated \ka for a given sequence, it had to compute a \ks score for each sequence of interest. All provable ensemble-based algorithms prior to \bbks, as well as many heuristic algorithms that optimize binding affinity, are \emph{single-sequence} algorithms which must compute the binding affinity for each possible sequence. The number of sequences, of course, is exponential in the number of simultaneously mutable residue positions.  Therefore, designs with many mutable residues rapidly became intractable. \osprey 3.0 provides a new algorithm, \bbks, which overcomes this challenge. \bbks~\cite{BBK*} builds on \ks, and is the first provable, ensemble-based protein design algorithm to run in time sublinear in the number of sequences. The key innovation in \bbks that enables this improvement is the \emph{multi-sequence (\msbound) bound}. Rather than compute binding affinity separately for each possible sequence, as single-sequence methods do, \bbks efficiently computes a single provable \ks score upper bound for a combinatorial number of sequences. \bbks uses \msbound bounds to prune a combinatorial number of sequences during the search, entirely avoiding single-sequence computation for all pruned sequences.

Importantly, \bbks also contains many other powerful algorithmic improvements and implementation optimizations: the parallel architecture of \bbks, which enables concurrent energy minimization, and a novel two-pass partition function bound, which minimizes far fewer conformations while still computing a provable $\varepsilon$-approximation to the partition function. Combined with the combinatorial pruning power of the \msbound bound, \bbks is able to search over much larger sequence spaces than previously possible with single-sequence \ks (Fig.~\ref{fig:bbks}). In computational experiments on 204 protein design problems, \bbks accurately predicted the tightest-binding sequences while only computing \ks scores for as few as one in 10$^5$ of the sequences in the search space~\cite{BBK*}.  Moreover, in computational experiments on 51 protein-ligand design problems, \bbks was up to 1982-fold faster than single-sequence \ks, despite provably producing the same results~\cite{BBK*}.  

These improvements show that \bbks not only accelerates protein designs that were possible with previous provable algorithms, it also efficiently performs designs that are too large for previous methods.

