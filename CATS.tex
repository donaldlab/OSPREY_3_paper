\subsection{CATS: Local backbone flexibility in all biophysically feasible dimensions}

{\sc osprey} pioneered protein design calculations that model local continuous flexibility of sidechains in the vicinity of rotamers in all biophysically feasible dimensions (i.e., the sidechain dihedrals).  This continuous flexibility was often critical in finding optimal sequences~\cite{iMinDEE}, and especially in eliminating artificial steric problems for ideal rotameric conformations that are chosen without consideration of protein context.  In {\sc osprey} 3, we now extend this ability to the backbone: allowing local continuous backbone flexibility in the vicinity of the native backbone in all biophysically feasible dimensions.  

This flexibility is enabled by the CATS algorithm~\cite{CATS}.  CATS uses a new parameterization of backbone conformational space, along with the voxel framework that {\sc osprey} has always included.  It is equivalent to searching over all changes in backbone dihedrals ($\phi$ and $\psi$) subject to keeping the protein conformation constant outside of a specified flexible region. CATS includes an efficient Taylor series-based algorithm for computing atomic coordinates from its new degrees of freedom, enabling efficient energy minimization.  Unlike previous protein design algorithms with backbone flexibility, CATS routinely finds backbone motions on the order of an angstrom while still performing a comprehensive search of its backbone conformation space.  In Ref.~\citen{CATS}, we have shown that backbone flexibility as modeled by CATS is sometimes critical for resolving artificial steric problems and often affects energetics significantly.  

CATS is intended to be run along with {\sc osprey}'s other algorithms, yielding efficient calculations with continuous flexibility in both the sidechains and the backbone. {\sc osprey}'s convenient interface allows a user to add CATS flexibility to a design merely by specifying the start and end points of the backbone segment to be made flexible.  