For over a decade, the {\sc osprey} software package~\cite{OSPREY,minDEE,OSPREY_MIE} has offered the protein design community a unique combination of continuous flexibility modeling, ensemble modeling, and algorithms with provable guarantees.  Having begun as a software release for the $K^*$ algorithm, which approximates binding constants using ensemble modeling, it now boasts a wide array of algorithms found in no other software.  {\sc osprey} has been used in many designs that were empirically successful---\textit{in vitro}~\cite{VRC07_enhance,CFTR,runx1_cbfb,GrsA-LeuA,DHFR-PNAS,GrsA-TyrA,specific_probes} and \textit{in vivo}~\cite{VRC07_enhance,CFTR,runx1_cbfb,DHFR-PNAS} as well as in non-human primates~\cite{VRC07_enhance}.  {\sc osprey}'s predictions have been validated by a wide range of experimental methods, including binding assays, enzyme kinetics and activity assays, in cell assays (MICs, fitness) and viral neutralization, {\em in vivo} studies, and crystal structures~\cite{DHFR-PNAS2, VRC07_enhance}.    

However, as we added more and more algorithms into {\sc osprey}, the code became somewhat complicated and messy.  Thus, we have now refactored it, to facilitate the adding of new features both by ourselves and by any others.  We have also introduced a convenient Python interface and GPU support, allowing designs to be completed much more quickly and easily than in previous version of {\sc osprey}.  We believe {\sc osprey} 3 will be a very useful tool for both developers and users of provably accurate protein design algorithms.  

\subsection{Past successes of {\sc osprey}}

{\sc osprey} has been used for an impressive number of empirically successful designs, ranging from enzyme design to antibody design to prediction of antibiotic resistance mutations.  Notably, {\sc osprey} has been successful in many~\textit{prospective} experimental studies, i.e., studies in which our designed sequences are tested experimentally, thus providing more realistic validation than a retrospective comparison of OSPREY calculations to previous experimental results.  {\sc osprey} is most applicable to problems that can be posed in terms of binding, allowing the $K^*$ algorithm and its variants to select the optimal sequence based on an estimate of binding free energy.  But most protein design problems can be posed in this way, sometimes in terms of binding to more than one ligand.  

For example, we have successfully predicted novel resistance mutations to new inhibitors in MRSA (methicillin-resistant~\textit{Staphylococcus aureus}), by searching for sequences that have impaired drug binding compared to wild-type DHFR, but still form the enzyme-substrate complex as usual, allowing catalysis~\cite{DHFR-PNAS,DHFR-PNAS2}.  Our predictions were validated not only biochemically and structurally, but also at an organismal level~\cite{DHFR-PNAS2, mimb_resistance}.  Similarly, we have successfully changed the preferred substrate of an enzyme---the phenylalanine adenylation domain of gramicidin S synthetase---from phenylalanine to leucine by modeling of the two enzyme-substrate complexes, searching for sequences with improved binding to leucine and less to phenylalanine.  

Still other successes of {\sc osprey} have involved improving a single binding interaction, like the interaction of the antibody VRC07 with its antigen, the gp120 surface protein of HIV.  Using this approach, we designed a broadly neutralizing antibody VRC07-523LS against HIV with unprecedented breadth and potency that is now in clinical trials~\cite{VRC07_enhance,clinical605}.  Likewise, we have used {\sc osprey} to develop peptide inhibitors of CAL, a protein involved in cystic fibrosis~\cite{CFTR}.  This is a protein design problem of direct therapeutic significance that consists of optimizing a protein-protein binding interaction.  

In addition, a number of other research groups have successfully used the {\sc osprey} algorithms and software (by themselves) to perform biomedically important protein designs, {\em e.g.,} to design anti-HIV antibodies that are easier to induce \cite{Georgiev:2014aa}; to design a soluble prefusion closed HIV-1-Env trimer with reduced CD4 affinity and improved immunogenicity~\cite{Gwo-yu17}; to design a transmembrane Zn$^{2+}$-transporting four-helix bundle~\cite{Joh14}; to optimize stability and immunogenicity of therapeutic proteins \cite{Parker:2013aa,Salvat:2015aa,Zhao:2015aa}; and to design sequence diversity in a virus panel and predict the epitope specificities of antibody responses to HIV-1 infection~\cite{polyclonal17}.

We believe {\sc osprey} 3 will enable an even greater range of successful designs.  
