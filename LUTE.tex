\subsection{LUTE: Putting advanced modeling into a form suitable for efficient, discrete design calculations}

{\sc osprey} 3 comes with LUTE~\cite{LUTE_RECOMB}, a new algorithm that addresses two issues with previous versions of {\sc osprey}.  

First, previous versions modeled continuous flexibility by enumerating conformations in order of a~\textit{lower bound} on minimized conformational energy~\cite{minDEE,iMinDEE}. LUTE addresses this problem by directly optimizing the minimized energies of full conformations, which are estimated using an expansion in low-order tuples of residue conformations.  Thus, the burden of modeling continuous flexibility is shifted from the combinatorial optimization (\as) step, which has unfavorable asymptotic scaling, to a precomputation step that only scales quadratically with the number of residues. 
Second, all previous combinatorial protein design algorithms have relied on an explicit decomposition of the energy as a sum of local (e.g., pairwise) terms.  This made design with energy functions that do not have this form difficult. LUTE can straightforwardly support general energy functions, and, as shown in~\cite{LUTE_RECOMB}, it can obtain good fits at least in the case of Poisson-Boltzmann energies.  

{\sc osprey} users can now turn on LUTE for continuously flexible calculations simply by setting a boolean flag.  %simply by setting the configuration ``useTupExp'' to true.  
{\sc osprey} 3 also supports design with Poisson-Boltzmann solvation energy calculations, which use the DelPhi~\cite{OSOR,DelPhi_surface} software for the single-point Poisson-Boltzmann calculations (we ask the user to download DelPhi separately for licensing reasons). Such improved modeling is essential to increasing the reliability of and range of feasible uses for computational protein design.  