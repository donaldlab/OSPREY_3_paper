
\begin{figure}
\caption{\textbf{The \osprey protein redesign suite.} (A) The input model includes a 3D structure of the protein to be redesigned, a definition of the sequence space, the allowed protein flexibility (including the rotamer library), and a pairwise energy function. (B) Rigid DEE~\cite{DEE,DEE/A*}, iMinDEE~\cite{iMinDEE}, EPIC~\cite{EPIC}, LUTE~\cite{LUTE_RECOMB}, DEEPer~\cite{DEEPer}, and CATS~\cite{CATS} model different types of protein flexibility. Flexibility ranges in complexity from discrete, rigid rotamers to continuous side chain flexibility to complete flexibility including continuous backbone flexibility. (C) The EPIC~\cite{EPIC} and LUTE~\cite{LUTE_RECOMB} algorithms also expand energy function capability by allowing for non-pairwise, basic quantum chemistry and Poisson-Boltzmann solvation. (D) COMETS~\cite{COMETS} allows for multi-state design by optimizing sequences and conformations for user-specified bound and unbound states. This is accomplished using multiple input structures. (E) These algorithms are implemented in OSPREY and improved through the use of GPU acceleration. According to the allowed flexibility, OSPREY runs a specific pruning algorithm followed by a highly optimized descendant of the \as search algorithm~\cite{dynamic_A*}. The \as output generates a ranking based on either the lowest-energy structure of each sequence, or an ensemble of structures computed by the \ks algorithm. (F) The \bwmstar~\cite{BWM*} algorithm exploits sparse residue interaction graphs and branch decomposition to outperform traditional \as. (G) The \ks~\cite{K*,minDEE} algorithm calculates a \ks score (an approximation of the binding constant, $K_a$) by provably estimating the partition function for the protein, the ligand, and the protein-ligand complex. The \ks algorithm exploits a thermodynamic ensemble of structures as opposed to a single structure, as illustrated in the panel (PDB ID: 3FQC). \ks can also be used to find sequences that have a high affinity for one ligand (positive design) while having a low affinity for another (negative design) by taking a ratio of \ks scores~\cite{DHFR-PNAS,DHFR-PNAS2}. }
\label{flowchart}
\end{figure}

\begin{figure}
\caption{\textbf{Runtimes of \osprey 2.2 vs. \osprey 3.0 for 45 protein design test cases (details shown in Table~\ref{table:2v3}), shown on a log scale.} Designs that only finished with \osprey 3.0 (given a 17-day time limit) are shown on the right in red.  All test cases involve continuous flexibility~\cite{minDEE,iMinDEE} and minimization-aware DEE~\cite{iMinDEE,EPIC}; 18 involve provably accurate partition function calculations (see Table~\ref{table:2v3} and Ref.~\citen{EPIC} for details).  }
\label{fig:2v3}
\end{figure}

\begin{figure}
\caption{\textbf{Benchmarks for protein conformation minimization in \osprey 3.0 for various hardware platforms and for conformations of varying size.} From smallest to largest: {\bf (top)} a single residue pair is the smallest multi-body minimization possible, {\bf (middle)} a full protein conformation with a single flexible residue represents a small design, {\bf (bottom)} a full protein conformation with 20 flexible residues represents a large design. For CPU hardware, concurrent minimizations correspond to CPU threads. For GPU hardware, concurrent minimizations correspond to {\it streams} defined by the CUDA framework. Faster minimization speeds correspond with faster \osprey runtimes. All minimizations were performed on the Atx1 metallochaperone protein (PDB ID: 1CC8)~\cite{1CC8}. Flexible residues were modeled with continuous sidechain flexibility, and all other residues remained completely fixed.}
\label{fig:gpu}
\end{figure}

\begin{figure}
\caption{\textbf{A Python script that performs a very simple design in \osprey 3.0.}  The design searches over sequences in which residues A2 and/or A3 of the Atx1 metallochaperone protein (PDB ID: 1CC8)~\cite{1CC8} are mutated; residues A2-A4 (i.e., residue 2-4 of chain A) are all modeled with sidechain flexibility, consisting of a discrete search over the Penultimate rotamer library\cite{penultimate}'s rotamers for the specified amino acid types.  The mutability, flexibility, and starting crystal structure are all specified in the ``define a strand'' section of the code.  Advanced users can also modify the other sections to specify changes from the default search algorithms, energy function, and other modeling assumptions.  This script uses the Max Product Linear Programming (MPLP) algorithm~\cite{MPLP} to reduce the size of the \as search tree~\cite{DEE/A*} employed for sequence and conformational search without compromising accuracy; see Ref.~\cite{dynamic_A*} for details.  }
\label{fig:pythonGMEC}
\end{figure}

\begin{figure}
\caption{\textbf{A Python script that performs a simple \bbks design in \osprey 3.0.}  This design produces a peptide to bind human fibronectin (the ``ligand strand,'' i.e. chain A) by optimizing a fragment of the protein FnBPA from~\textit{Staphylococcus aureus} (the ``protein strand,'' chain G), which has been crystallized in complex with fibronectin domains (PDB ID: 2RL0~\cite{2RL0}).  As in Fig.~\ref{fig:pythonGMEC}, the script defines the starting crystal structure, mutable residues, and level of mutability and flexibility (here including continuous flexibility) in the form of Python strand objects.   Fig.~\ref{fig:pythonBBKSpic} represents this design graphically.  This design is accelerated by parallelism, running on 4 CPU cores.  This example thus shows it is easy to invoke and use parallelism within the \osprey 3.0 software.  }
\label{fig:pythonBBKS}
\end{figure}

\begin{figure}
\caption{\textbf{Setup for the Python-scripted \bbks~\cite{BBK*} design described in Fig.~\ref{fig:pythonBBKS}.}  This design starts with the crystal structure (PDB ID: 2RL0~\cite{2RL0}) of a complex between fragments of the protein FnBPA from~\textit{Staphylococcus aureus} (blue ribbons) and human fibronectin (green ribbons), and optimizes binding with respect to the amino acid type of FnBPA residue 649 (magenta), while modeling continuous flexibility in several surrounding sidechains (orange).  The full complex is shown on the left, while the region surrounding the mutation is shown in detail on the right.  See Ref.~\citen{BBK*} for background on the FnBPA:fibronectin system.  }
\label{fig:pythonBBKSpic}
\end{figure}

\begin{figure}
\caption{Left: CATS allows systematic search over a voxel of backbone conformations in the vicinity of the wild-type backbone conformation (black).  The voxel is specified as box constraints on a novel set of backbone coordinates; conformations with one such coordinate moved to the edge of the voxel are shown in red and green, and a conformation with all such coordinates moved to the edge of the voxel is shown in purple.  See Fig. 1 of Ref.~\citen{CATS} for more details.  Middle: Rigid-backbone structural modeling of an experimentally effective mutant of anti-HIV gp120 antibody VRC07 showed unavoidable steric clashes between Trp 54 of VRC07 and Trp 427 and Gly 473 of gp120 (purple).  Right: CATS explained the experimentally observed activity by finding a new backbone conformation that resolved these clashes (green; overlaid with clashing rigid-body backbone (purple) and backbone conformation computed with the older DEEPer algorithm (blue)).  DEEPer reduced the clashes somewhat using backrub motions~\cite{backrub}, but they were still significant even after the backrubs.  See Fig. 3 of Ref.~\citen{CATS} for more details.   Portions of this figure were reprinted with permission from Ref.~\citen{CATS}.  }
\label{fig:cats}
\end{figure}


\begin{figure}
\caption{\textbf{Measuring the contributions of the \bbks~\cite{BBK*} algorithmic improvements to the empirically observed running times of \osprey.}  \bbks calculations were run to predict either the single top sequence or to enumerate the top 5, and were compared to exhaustive computation of \ks scores for each sequence (i.e., iMinDEE/\as/\ks~\cite{minDEE,iMinDEE} or single-sequence \ks), which was the prior state of the art for Boltzmann-weighted ensemble-based binding affinity computation before \bbks.  (A) Running times for \bbks and single-sequence \ks vs.~the number of sequences in the search space for 204 protein design test cases, a benchmark set described in Ref.~\citen{BBK*}.  Single-sequence \ks completed only 107 of the test cases within a 30-day time limit (left of the vertical line), and took up to 800 times longer than \bbks to do so, while \bbks completed all the designs within the time limit.  (B) The number $N$ of sequences whose energies must be examined or bounded by iMinDEE/\as/\ks (green line; exponential in the number of mutable residue positions) and by \bbks (dots). For each data point representing a \bbks test case, the vertical gap between that data point and the green line (gap on the $y$ axis) represents the number of sequences that are pruned without ever having to be examined.  Figure adapted with permission from Ref.~\citen{BBK*}.  }
\label{fig:bbks}
\end{figure}


\begin{figure}
 \caption{(A) The structure of the IFN$\alpha$2:IFNAR2 complex (PDB ID: 3S9D \cite{pdb3s9d}) with separate chains shown in cyan and magenta and with two example interface design regions shown in boxes. Each box contains a mutable residue shown as sticks and its surrounding flexible residues shown as lines. (B) and (F) zoom in on each design. (B-E) Design at position R33 for a mutation that \osprey correctly predicts as decreasing binding: R33Q. (B) The wildtype sequence with probe dots \cite{Probe,PIV} displaying favorable interactions with surrounding flexible residues (shown as lines). (D) The mutant sequence (33Q) with probe dots displaying some favorable as well as unfavorable interactions. Comparing (B) and (D), it is clear there is a loss in favorable interactions and a gain in unfavorable interactions upon mutation from R to Q, resulting in an experimentally observed decrease in binding that the \ks algorithm captures accurately (See Table \ref{table:spearman}). (C) and (E) show the top 10 conformations in the conformational ensemble used in the \ks calculation for each sequence. (F-I) Design at position N156 for a mutation that \osprey correctly predicts as increasing binding: N156A. (F) The wildtype sequence with probe dots~\cite{Probe,PIV} displaying some favorable interactions with surrounding flexible residues (shown as lines). (H) The mutant sequence (156A) with probe dots displaying some favorable interactions with surrounding flexible residues (shown as lines). There are some gained interactions (shown by an increase in the number of favorable probe dots) in (H) compared to (F), but these are not visually obvious, thus emphasizing the importance of \ks, which successfully picks up these nuanced changes and correctly predicts improved binding (See Table \ref{table:spearman}). (G) and (I) show the top 10 conformations in the conformational ensemble used in the \ks calculation for each sequence. Not shown are the ensembles for the unbound states that are also used to calculate the \ks scores.}
\label{fig:designs}
\end{figure}

\begin{figure}
\caption{Testing the accuracy of the \ks algorithm in \osprey 3.0 by comparing \ks rankings to experimentally reported rankings (See Table \ref{table:spearman}). Each system is represented by its corresponding PDB ID and a linear trendline is shown for each in its corresponding color according to the legend.}
\label{fig:rankings}
\end{figure}


